
\documentclass[11pt]{article}

% Set the margins to be something normal.
\usepackage[margin=1in]{geometry}

% Fun lists!
\usepackage{enumitem}
\usepackage{textcomp}
\setitemize{label=\textrightarrow, itemsep=0pt}

% Math symbols.
\usepackage{amsmath, amssymb, amsfonts}

% No indents.
\setlength\parindent{0pt}

% Nice fonts :)
\usepackage{tgschola}

\begin{document}
	{
		\centering
		\huge{Week 5 Recitation Problems} \\
		\Large{MATH:114, Recitations 309 and 310} \\
	}
	\vspace{3em}
	
	{
		\centering
		\large{\textbf{Curve Length and Surface Area}} \\
	}
	
	\vspace{3em}
	1. Given a function $f(x)$, how might we \textbf{approximate} the length of $f(x)$ on the closed interval $[a,b]$? Draw an annotated picture or write a few words to explain, and include relevant geometric formulas or ideas. \textit{(Hint 1: use the Euclidean distance formula, which you are free to look up. Hint 2: break the curve up into chunks!)}
	
	\vspace{0.3\textheight}
	2. Using your strategy from Problem 1, translate your approximation into an exact continuous calculation (that is, one which uses an integral). Draw an annotated picture or write a few words to explain, and include relevant calculus theorems or geometric ideas. \textit{(Hint: think about the rectangle or trapezoid methods for estimating the area under a curve, which you are free to look up.)}
	
	\newpage
	3. Let $$f(x)=\frac{x^3}{6} + \frac{1}{2x}.$$ Find the length of $f(x)$ when $1 \leq x \leq 3$.
	
	\vspace{0.2\textheight}
	4. The formula $$ S = 2\pi \int_a^b g(x) \cdot \sqrt{1 + g'(x)^2} \ dx $$ describes how to find the surface area of the solid generated by the curve $g(x)$ on the closed interval $[a,b]$. What is familiar about this formula? Using annotated pictures or a few words, describe the geometric ideas at work here.
	
	\vspace{0.3\textheight}
	5. Let $g(x)=\sqrt{4-x^2}$, and $-1 \leq x \leq 1$. Find the surface area of the solid generated by rotating $g(x)$ around the $x$ axis.
	
\end{document}
