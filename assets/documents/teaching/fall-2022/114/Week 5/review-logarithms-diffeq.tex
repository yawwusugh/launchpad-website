
\documentclass[11pt]{article}

% Set the margins to be something normal.
\usepackage[margin=1in]{geometry}

% Fun lists!
\usepackage{enumitem}
\usepackage{textcomp}
\setitemize{label=\textrightarrow, itemsep=0pt}

% Math symbols.
\usepackage{amsmath, amssymb, amsfonts, nicefrac}

% No indents.
\setlength\parindent{0pt}

% Nice fonts :)
\usepackage{tgschola}

\begin{document}
	{
		\centering
		\huge{Week 5 Recitation Problems} \\
		\Large{MATH:114, Recitations 309 and 310} \\
	}
	\vspace{3em}
	
	{
		\centering
		\large{\textbf{Logarithms and Exponential Change}} \\
	}
	\vspace{3em}
	
	1. Evaluate $$ \int \frac{1}{x \ln(x)} \ dx $$ 
	
	\vspace{0.3\textheight}
	2. Evaluate $$ \int \frac{x}{x^2+4} \ dx $$
	
	\newpage
	If a function $y(t)$ is increasing or decreasing at an exponential rate, we can say it is \textbf{exponentially growing} or \textbf{exponentially decreasing}, and this rate of change is proportional to its value at a time $t$. In other words, $y(t)$ is \textbf{proportional to its own derivative} $y'(t)$, so \begin{equation*} \frac{d}{dt} \ y(t) = k \cdot y(t). \tag{$\ast$}\end{equation*} Writing this just in terms of our function $y$, and treating it like a variable, the following expressions are equivalent: $$ \frac{dy}{dt} = k \cdot y(t) \hspace{3em} y'(t) = k \cdot y(t) \hspace{3em}  y' = k\cdot y$$
	A \textbf{differential equation} is when a function is equated to its own derivative(s), in an expression like the ones above. An \textbf{initial value problem} arises when you are given $k$ and the value of $y(t)$ at a ``starting value'' or \textbf{initial condition} $t_0$ on its domain --- like $t_0=0$, so $y(t_0) = y(0) = C$, where $C$ is some constant --- and we are tasked with recovering the function $y(t)$. The above types of initial value problems have one specific solution: $$ y(t) = Ce^{kt}, $$ where $k>0$. 
	
	\vspace{3em}
	3. Check that the function $y(t) = Ce^{kt}$ satisfies the equation in ($\ast$).
	
	\vspace{0.1\textheight}
	4. Find the general solution for the initial value problem where $k=\nicefrac 14$ and $y(t_0) = y(0) = 200$.
	
	\vspace{0.2\textheight}
	5. If $y(t)$ from Problem 4 describes the population of mosquitoes, when will we triumph over our pestilent insect overlords and vanquish their population? (In other words, at what time $t$ does $y(t) = 0$?)	
\end{document}
