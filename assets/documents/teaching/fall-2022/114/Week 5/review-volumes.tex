
\documentclass[11pt]{article}

% Set the margins to be something normal.
\usepackage[margin=1in]{geometry}

% Fun lists!
\usepackage{enumitem}
\usepackage{textcomp}
\setitemize{label=\textrightarrow, itemsep=0pt}

% Math symbols.
\usepackage{amsmath, amssymb, amsfonts}

% No indents.
\setlength\parindent{0pt}

% Nice fonts :)
\usepackage{tgschola}

\begin{document}
	{
		\centering
		\huge{Week 5 Recitation Problems} \\
		\Large{MATH:114, Recitations 309 and 310} \\
	}
	\vspace{3em}
	
	{
		\centering
		\large{\textbf{Volumes}} \\
	}
	
	\vspace{3em}
	
	1. Suppose the functions $f(x)$ and $g(x)$ bound a closed region $R$ in the plane. Rotate $R$ around the $x$ axis to get a solid of rotation $S_R$. How does the \textbf{washer} method find the volume of $S_R$? Use words or pictures to explain, including relevant geometric formulas or ideas.
		
	\vspace{0.35\textheight}
	2. Let $f(x) = x^2$ and $g(x)=x+2$, and let $R$ be the closed region bounded by $f(x)$ and $g(x)$. Find the volume of the solid generated by rotating $R$ around the $x$ axis.
	
	\newpage
	3. Let the functions $p(x)$ and $q(x)$ bound a closed region $C$ in the plane. Rotate $C$ around the $x$ axis to get a solid of rotation $S_C$. How does the \textbf{shell} method find the volume of $S_C$? Use words or pictures to explain, including relevant geometric formulas.
	
	\vspace{0.3\textheight}
	4. Why might it be difficult to use the shell method with the functions $f(x)$ and $g(x)$ from Problem 2? \textit{(Hint: how do we find the inverse of $f(x)$?)}
	
	\vspace{0.2\textheight}
	5. Let $p(x)=x^2$ and $q(x)=-x^4$. Set up an integral to find the volume of the solid found by rotating the region bounded by $p(x)$, $q(x)$, and the vertical line $x=1$ around the $y$ axis. If you have time, compute this integral!
	
\end{document}
