
\documentclass[12pt]{article}

% Set the margins to be something normal.
\usepackage[margin=1in]{geometry}

% Fun lists!
\usepackage{enumitem}
\usepackage{textcomp}
\setitemize{label=\textrightarrow, itemsep=0pt}

% Math symbols.
\usepackage{amsmath, amssymb, amsfonts}

% No indents.
\setlength\parindent{0pt}

% Nice fonts :)
\usepackage{tgschola}

\begin{document}
	\thispagestyle{empty}

	{
		\centering
		\huge{Week 1 Recitation Problems} \\
		\Large{MATH:114, Recitations 309 and 310} \\
	}
	
	\vspace{3em}
	1. Graph (shade) the region bounded by the following curves in the first quadrant:
	
	\begin{enumerate}[label=(\alph*)]
		\item $y = f(x) = (x+1)^2 + 1$
		\item $y = g(x) = 9-4x$
		\item The $y$-axis.
	\end{enumerate}
	
	\vspace{12em}
	
	2. Write one (or two) $x$-integrals that give the exact area of this region. Using your integral(s), compute this area.
	
	\vspace{12em}
	
	3. Write one (or two) $y$-integrals that give the exact area of this region. Using your integral(s), compute this area.
	
	\newpage
	
	Let's check our work by answering some questions:
	
	\begin{enumerate}
		\item Are your results for questions 2 and 3 the same?
		\item Can you find a way to \textit{approximate} the area between the curves? (Hint: use a bit of geometry!)
		\item How does your approximation compare to the values you computed in (2) and (3)?	
	\end{enumerate}
	
	\vspace{12em}
	
	4. Repeat the same process with these curves:
	
	\begin{enumerate}
		\item $f(x)=x^3$
		\item $g(x)=2-x^2$
		\item the $x$-axis.
	\end{enumerate}

	
\end{document}
