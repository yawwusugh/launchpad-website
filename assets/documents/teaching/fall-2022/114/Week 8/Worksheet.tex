
\documentclass[11pt]{article}

% Set the margins to be something normal.
\usepackage[margin=1in]{geometry}

% Fun lists!
\usepackage{enumitem}
\usepackage{textcomp}
\setitemize{label=\textrightarrow, itemsep=0pt}

% Math symbols.
\usepackage{amsmath, amssymb, amsfonts, amsthm, nicefrac}
\newtheorem*{theorem*}{Theorem}

\usepackage{array}
    \newcolumntype{P}[1]{>{\centering\arraybackslash}p{#1}}
    \newcolumntype{M}[1]{>{\centering\arraybackslash}m{#1}}

% No indents.
\setlength\parindent{0pt}
\usepackage{changepage}

% Nice fonts :)
\usepackage{tgschola}

\begin{document}
	\thispagestyle{empty}
	{
		\centering
		\huge{Week 8 Recitation Problems} \\
		\Large{MATH:114, Recitations 309 and 310} \\[2em]
	}
	\normalsize{Names:} \hrulefill
	\vspace{3em}
	
	Last week, we covered \textbf{linear} and \textbf{quadratic approximations} for a given function $f$. These approximations lead to \textbf{Taylor's Theorem}, which says:
	
	\begin{theorem*}[Taylor]
		Let $f$ be continuously differentiable $N+1$ times at the point $a$. Then, there is a function $R(x)$ and a point $c$ between $a$ and $x$ which satisfies the following equation: $$ f(x) = P_N(x) + R_N(x),$$ where $$P_N(x) = \underbrace{f(a) + f'(a)(x-a) + \frac{f''(a)}{2}(x-a)^2}_{\text{quadratic approximation!}} + \cdots + \frac{f^N(a)}{N!}(x-a)^N, $$ and $$ R(x) = \frac{f^{N+1}(c)}{(N+1)!}(x-a)^{N+1}.$$
	\end{theorem*}
	
	The function $P_N(x)$ is the $N$\textsuperscript{th}-order \textbf{Taylor polynomial} --- that is, a polynomial of degree $N$ which approximates $f$ at $a$. $R_N(x)$ is the \textbf{remainder} or \textbf{error} function, and represents how far away $P_N(x)$ is from $f(x)$.
	
%	\vspace{1em}
%	\begin{adjustwidth*}{0.5in}{0.5in}
%		\textbf{The big picture:} if $f(x)$ has at least $N+1$ derivatives at $a$, we can \textit{exactly compute} $f(x)$ using a Taylor polynomial and an error term.
%	\end{adjustwidth*}
	
	
	
	\vspace{3em}
	1. Let $f(x) = \sin(x)$ and $a=0$. Compute the $6$\textsuperscript{th}-order Taylor polynomial $P_6(x)$ and the remainder function $R_6(x)$. What pattern do you see?
	
	\vspace{0.15\textheight}
	2. Suppose $f(x) = \sin(x)$ and $a=0$, and that $N$ is an arbitrary finite number. Write an expression for the $N$\textsuperscript{th}-order Taylor polynomial $P_N(x)$ of $f(x)$ using summation notation --- that is, $$ P_N(x) = \sum_{k=1}^N \frac{f^k(a)}{k!}x^k$$ Also find the remainder function $R_N(x)$. Use the pattern you found in Problem 1 to help. \textbf{Once you finish this question, STOP and tell one of the instructors. We'll discuss our results as a class before moving on!}
	
	\newpage
	
	3. Fill out the following table. \textbf{In the first column}, write one of the four expressions for $R_N(x)$ that we discussed as a class. \textbf{In the second column}, write the minimum and maximum possible value of the $f^{N+1}(c)$ that appears in $R_N(x)$. \textbf{In the third column}, write the absolute value of the value in the previous column. \textbf{In the fourth column}, substitute the absolute value in the previous question for $f^{N+1}(c)$ in the expression of $R_N(x)$. The first row of the table is filled out for you, with $N = 2k+1$.

	\vspace{2em}
	
	\begin{table}[h]
		\centering
		\renewcommand{\arraystretch}{3}
		\begin{tabular}{c|c|c|c}
			\ \ \ $R_N(x)$ \ \ \ \ & \ \ \ $\min, \max f^{N+1}(c)$ \ \ \ & \ \ \ $|\min, \max f^{N+1}(c)|$ \ \ \ & \ \ \ $R_N(x)$ Upper Bound \ \ \ \\ \hline
			
			$R_N(x) = \frac{\sin(c)}{(2k+2)!}x^{2k+2}$ & $-1, 1$ & $1$ & $R_N(x) \leq \frac{(1)}{(2k+2)!}x^{2k+2} = \frac{x^{2k+2}}{(2k+2)!}$ \\ \hline
			& & & \\ \hline
			& & & \\ \hline
			& & & \\ \hline
		\end{tabular}
	\end{table}
	
	\vspace{1em}
	What do you notice about the upper bounds of $R_N(x)$?
	
	\vspace{2em}
	4. Using a a visualizer like Desmos, graph the function $$ \frac{\left(\frac \pi 2\right)^{x+1}}{(x+1)!} $$ Using this graph, what can you say about the function $$ \frac{x^{N+1}}{(N+1)!}$$ as $N$ goes to infinity? \textbf{Once you finish this question, STOP and tell one of the instructors. We'll discuss our results as a class!}
	
	\vspace{2em}
	5. Based on our discussion, what can we say about the remainder $R_N(x)$ as $N$ goes to infinity? Use this to conclude that \vspace{2em}$$ \hspace{-0.5\textwidth} \sin(x) = $$


	
\end{document}
