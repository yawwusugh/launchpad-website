
\documentclass[11pt]{article}

% Set the margins to be something normal.
\usepackage[margin=1in]{geometry}

% Fun lists!
\usepackage{enumitem}
\usepackage{textcomp}
\setitemize{label=\textrightarrow, itemsep=0pt}

% Math symbols.
\usepackage{amsmath, amssymb, amsfonts}

% No indents.
\setlength\parindent{0pt}

% Nice fonts :)
\usepackage{tgschola}

% Nice fractions!
\usepackage{nicefrac}
\usepackage{array,multirow,graphicx}
\usepackage{float}

\newcommand{\STAB}[1]{\begin{tabular}{@{}c@{}}#1\end{tabular}}

\begin{document}
	{
		\centering
		\huge{Week 9 Recitation Problems} \\
		\Large{MATH:114, Recitations 309 and 310} \\[2em]
	}
	\normalsize{Names:} \hrulefill
	\vspace{3em}
	
	An \textbf{improper integral} is the definite integral of a function where \textbf{one or both of the limits of integration approach infinity} or \textbf{the function is discontinuous somewhere on the interval of integration.}
	
	\vspace{0.05\textheight}
	
	1. Using what you've covered in lecture, fill out the table below. How are you thinking about these answers?
	
	\begin{table}[h]
		\centering
		\renewcommand{\arraystretch}{3}
		\begin{tabular}{|c|c|c|} \hline
			$p=$ & Integral & Finite or Infinite? \\ \hline
			$1$ & $ \displaystyle \int_1^\infty \frac 1x \ dx$ & \\ \hline
			$\nicefrac 12$ & $ \displaystyle \int_1^\infty \frac{1}{\sqrt x} \ dx$ & \\ \hline
			$2$ & $ \displaystyle \int_1^\infty \frac{1}{x^2} \ dx$ & \\ \hline
		\end{tabular}
	\end{table}
	
	\vspace{0.05\textheight}
	2. If $p \neq 1$, compute the improper integral $$ \int_1^\infty \frac{1}{x^p} \ dx. $$ What are the conditions on $p$ that determine the convergence of the integral? Why do they make sense?
	
	\vspace{0.1\textheight}
	\dotfill
	\vspace{2em}
	
	Let's find out if $ \int_3^\infty \nicefrac{\ln(x)}{\sqrt x} \ dx $ is convergent.
	
	\vspace{0.025\textheight}
	3. Draw and compare the graphs of $f(x) = \ln(x)$ and $g(x) = 1$. When $x \geq 3$, which of the functions is greater than the other?
	
	\vspace{0.1\textheight}
	
	4. Using the result from Problem 4, what can you say about the functions $ \nicefrac{\ln(x)}{\sqrt x}$ and $\nicefrac{1}{\sqrt x}$ when $x \geq 3$? If we integrate them as $ \int_3^\infty \nicefrac{\ln(x)}{\sqrt x} \ dx$ and $\int_3^\infty \nicefrac{1}{\sqrt x} \ dx,$ which integral should be bigger?
	
	\vspace{0.1\textheight}
	5. Compute the integral $\int_3^\infty \nicefrac{1}{\sqrt x} \ dx$. Based on your result, what can you say about $ \int_3^\infty \nicefrac{\ln(x)}{\sqrt x} \ dx$?
	
	\vspace{0.1\textheight}
	
	6. Suppose we have two functions $f(x)$ and $g(x)$, and let $f(x) \geq g(x) \geq 0$ where $x \geq a$. If...
	
		\begin{enumerate}[label=,itemindent=0.19\linewidth, itemsep=2pt]
			\item $\int_a^\infty f(x) \ dx$ diverges \hspace{0.3em} $ \implies \int_a^\infty g(x) \ dx$ \rule{3cm}{0.15mm}
			\item $\int_a^\infty f(x) \ dx$ converges $\implies \int_a^\infty g(x) \ dx$ \rule{3cm}{0.15mm}
			\item $\int_a^\infty g(x) \ dx$ diverges \hspace{0.4em} $ \implies \int_a^\infty f(x) \ dx$ \rule{3cm}{0.15mm}
			\item $\int_a^\infty g(x) \ dx$ converges $ \implies \int_a^\infty f(x) \ dx$ \rule{3cm}{0.15mm}
		\end{enumerate}
		\textit{Note: the symbolic phrase $a \implies b$ means ``if a, then b.''}
		
	\vspace{1em}
	\dotfill
	\vspace{2em}
	
	7. Think about the integral $$ \int_2^\infty \frac{\cos^2(t)}{t^2} \ dt. $$ Do you think that this integral converges or diverges? Why?
	
	\vspace{0.1\textheight}
	8. What is a good function to compare the above integrand to? Write an inequality to justify your answer.

	
\end{document}
