
\documentclass[10 pt]{article}
%\usepackage{gmudissertation}
\usepackage{amsthm}         
\usepackage{graphicx}                    %   for imported graphics
\usepackage{amsmath}                     %%
\usepackage{amsfonts}                    %%  for AMS mathematics
\usepackage{amssymb}                     %%             
\usepackage[normalem]{ulem}              %   a nice standard underline package
\usepackage[noadjust,verbose,sort]{cite} %   arranges reference citations neatly
\usepackage{setspace}                    %   for line spacing commands
\usepackage{lmodern}
\usepackage{verbatim}
\usepackage{bm}
\usepackage{comment}
\usepackage{wrapfig}
\usepackage{lipsum}
\usepackage{tikz}
\usepackage{mathtools}
\usepackage{mhchem}
%\usepackage{cite}

\newcommand{\verteq}{\rotatebox{90}{$\,=$}}
\newcommand{\equalto}[2]{\underset{\scriptstyle\overset{\mkern4mu\verteq}{#2}}{#1}}

\addtolength{\oddsidemargin}{-1.1in}
\addtolength{\evensidemargin}{-1.5in}
\addtolength{\textwidth}{2.3in}

\addtolength{\topmargin}{-1.1in}
\addtolength{\textheight}{2in}
\begin{document}
{\Large\bf{THE GREATEST THING YOU WILL DO ALL WEEK\#11 }}
\begin{center}
	\Large{  MATH 114 - CALCULUS II -  SPRING 2020}
\end{center}
{ \bf Professor/TA :} \underline{\hspace{ 5cm}} \hspace{1 cm} {\bf Sec:} \underline{\hspace{ 4.7cm}}
\vspace{0.3 cm}\\
{\bf FULL NAME:} \ \underline{\hspace{ 4.9cm}} \hspace{ 1 cm}
{\bf Partners:} \underline{\hspace{ 4cm}}
\vspace{1 cm}\\


\noindent\makebox[\linewidth]{\rule{\textwidth}{0.01pt}}
\textbf{Approximating functions using polynomials.}\\
\noindent\makebox[\linewidth]{\rule{\textwidth}{0.01pt}}

\begin{itemize}
	\item[(A)] Let us first do LINEAR approximations. We want to approximate the function $g(x) = e^x$ about $x = 0$.
\end{itemize}
	\begin{minipage}{0.55\textwidth}
	\begin{itemize}
\item[1)] Deliberate with your group about how you might approximate the function $g(x) = e^x$ about $x = 0$. Explain your thinking.
\vspace{ 3 cm}
\item[(2)] Use one of the discussed methods to approximate the
function using a line $L(x) = a_0 + a_1x$. Draw the linear
approximation $L(x)$ on the graph.
%\vspace{ 2 cm}
\end{itemize}
\end{minipage}
\begin{minipage}{0.7 \textwidth}
	\includegraphics[scale=0.6]{12_Graph_1.png}
\end{minipage}
\begin{itemize}
	\vspace{ 2 cm}
\item[(3)]Find a formula for the tangent line of an arbitrary function $f(x)$ about at the point $x = 0$.
\vspace{ 3 cm}
\end{itemize}
\begin{itemize}
	\item[(B)] Let us now tackle QUADRATIC approximations.
	\begin{itemize}
		\item[1)] Let us find the quadratic $Q(x) = a_0 + a_1 x + a_2 x^2$ that approximates the function $g(x) = e^x$ around $x = 0$.
		\end{itemize}
\begin{itemize}
\item[(i)] Match function values:\\
\includegraphics[scale=0.5]{12_Pic_1.png}
%\tikz[baseline=0.6ex]\draw (0,0) rectangle (2cm,3ex) \tikz[baseline=0.6ex]\draw (0,0) rectangle (2cm,3ex)
\item[(ii)] Match tangents:\\
\includegraphics[scale=0.5]{12_Pic_2.png}
\item[(iii)] Match concavity:\\
\includegraphics[scale=0.5]{12_Pic_3.png}
\end{itemize}

\begin{minipage}{0.5 \textwidth}
\begin{itemize}
\item[2)] Finally, $Q(x)=$ \tikz[baseline=0.6ex]\draw (0,0) rectangle (5 cm,4ex);\\

Use these results to graph the quadratic Q(x) (make
sure to match location, slope and concavity!).
\item[3)] Find a formula for the quadratic approximation of an arbitrary
function $f(x)$ about the point $x=0$.
\vspace{ 2 cm}
	\end{itemize}
\end{minipage}
\begin{minipage}{0.7 \textwidth}
	\includegraphics[scale=0.6]{12_Graph_1.png}
\end{minipage}
(C) Complete the following table. Which method provides the best approximation? Why?\\

\begin{tabular}{|p{5.5 cm}|p{5.5 cm}|p{5.5 cm}|}
	\hline
	\hspace{ 2 cm}Functions  & \hspace{ 1.5 cm} Value at $x=1$ & \hspace{ 2 cm}Error \\
	\hline
	$g(x)=e^x$ & \hspace{ 2 cm} $2.71828$ &  \\
	\hline
	$L(x)=$ & & \\
	\hline
	$Q(x)= $ & & \\
	\hline 
\end{tabular}
\vspace{0.1 cm}\\

(D) Find a cubic $[C(x)=a_0+a_1x+a_2x^2+a_3x^3]$ approximation for $g(x)=e^x$ at $x=0$. \\
\vspace{ 4 cm}

(E) Extra time: Discuss how you could generalize this to a polynomial of degree $n$ $\bigg[
P(x)=\sum\limits_{i=0}^na_ix_i\bigg]$ approximation at $x=0$.
\end{itemize}
\end{document}