
\documentclass[11pt]{article}

% Set the margins to be something normal.
\usepackage[margin=1in]{geometry}

% Fun lists!
\usepackage{enumitem}
\usepackage{textcomp}
\setitemize{label=\textrightarrow, itemsep=0pt}

% Math symbols.
\usepackage{amsmath, amssymb, amsfonts}

% No indents.
\setlength\parindent{0pt}
\setlength\parskip{1em}

% Nice fonts :)
\usepackage{tgschola}

\begin{document}
	\thispagestyle{empty}
	{
		\centering
		\huge{Week 5 Recitation Problems} \\
		\Large{MATH:114, Recitations 309 and 310} \\
	}
	\vspace{3em}
	
	First, let's talk about \textbf{linear approximations.}
	\vspace{3em}
	
	1. Let's approximate the function $f(x)=e^x$ around $x=0$. Discuss a few ideas for this specific approximation.
	
	\vspace{0.2\textheight}
	
	2. Let $L(x) = a_0 + a_1x$. What kind of function does $L(x)$ describe? Try to approximate $f(x)$ at $x=0$ by adjusting $a_0$ and $a_1$, and draw a picture of your approximation.
	
	\vspace{0.2\textheight}
	
	3. Come up with a general-purpose formula for approximating an arbitrary function $g(x)$ near the point $x$. \textit{(Hint: what does the Mean Value Theorem say?)}
	
	\newpage
	Now, we can talk about \textbf{quadratic} and \textbf{higher-order} approximations. To do so, we're going to find a quadratic function $Q(x) = a_0 + a_1x + a_2x^2$ that approximates $f(x)$ near the point $x$. From the previous page, let $$f(x) = e^x.$$
	
	4. To construct $Q(x)$, we want the first \textit{and} second derivatives to look a lot like the first and second derivatives of $f(x)$. In each of the following equations, set $f(x)$ (or its derivatives) equal to $Q(x)$ (or its derivatives) and solve for each coefficient.
	
	First, match the values of the functions: $$ f(0) = Q(0) \implies a_0 = \rule{1cm}{0.15mm} $$ Then, match the values of the first derivatives: $$ f'(0) = Q'(0) \implies a_1 = \rule{1cm}{0.15mm}$$ Finally, match the values of the second derivatives: $$ f''(0) = Q''(0) \implies a_2 = \rule{1cm}{0.15mm}$$
	
	\vspace{3em}
	5. Using the coefficients you just found, write out $Q(x)$. Can you come up with a general-purpose formula for the quadratic approximation $Q(x)$ for an arbitrary function $g(x)$? What about a cubic approximation?
	
	\vspace{0.2\textheight}
	6. Find the cubic approximation $C(x)$ for $f(x) = e^x$. For each of the linear, quadratic, and cubic approximations, check its value against the true value of $f(x)$ at $x=0$.
\end{document}
