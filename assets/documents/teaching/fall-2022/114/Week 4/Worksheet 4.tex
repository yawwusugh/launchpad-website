
\documentclass[11pt]{article}

% Set the margins to be something normal.
\usepackage[margin=1in]{geometry}

% Fun lists!
\usepackage{enumitem}
\usepackage{textcomp}
\setitemize{label=\textrightarrow, itemsep=0pt}

% Math symbols.
\usepackage{amsmath, amssymb, amsfonts}

% No indents.
\setlength\parindent{0pt}

% Nice fonts :)
\usepackage{tgschola}

\begin{document}
	\thispagestyle{empty}
	{
		\centering
		\huge{Week 3 Recitation Problems} \\
		\Large{MATH:114, Recitations 309 and 310} \\
	}
	\vspace{3em}
	
	
Let's sketch the graphs of the hyperbolic functions.
\[\cosh (x)=\frac{e^{x}+e^{-x}}{2} \quad \text { and } \quad \sinh (x)=\frac{e^{x}-e^{-x}}{2}.\]
\textbf{(1)} Sketch the functions: $f(x)=\frac{1}{2} e^{x} \text { and } g(x)=\frac{1}{2} e^{-x}$. \ \textbf{(2)} Sketch the functions: $f(x)=\frac{1}{2} e^{x} \text { and } h(x)=-\frac{1}{2} e^{-x}$.
\vspace{6 cm}\\
\textbf{(3)} On the same graph as (1) above sketch $\cosh(x)$. Also, on same graph as (2) above sketch $\sinh(x)$. Comment on the limits when $x$ tends to $-\infty$ and $+\infty$ for $\cosh(x)$ and $\sinh(x)$. Is this consistent with your graphs?
\vspace{3 cm}\\ 
Are $\cosh(x)$ and $\sinh(x)$ odd or even functions? How do you know?

\newpage
\textbf{(4)} The names of these hyperbolic functions are very similar to that of $\cos(x)$ and $\sin(x)$, which is no accident.
Explore their similarities and differences by filling out the following tables.
\vspace{0.5  cm}\\
\begin{tabular}{|c|p{4 cm} |}
	\hline
	\textbf{Function} & \textbf{Derivative} \\
	\hline
	$\sin(x)$ & \\
	\hline
	$\cos(x) $ & \\
	\hline
	$\sinh(x)$ & \\
	\hline
	$\cosh(x) $ & \\
	\hline
\end{tabular}
\hspace{2 cm}
	\begin{tabular}{|p{6 cm}|}
		\hline
	\textbf{Identities}\\
	\hline
	$\cos^2(x)+\sin^2(x)=$\\
	\hline
	$\cosh^2(x)-\sinh^2(x)=$\\
	\hline
\end{tabular}
\vspace{3 cm}\\
\textbf{(5)} Application: The shape of a hanging cable between two same-
height poles satisfies the equation $y = A \cosh(x/C) + B$. Assume
that the cable's tension in our example is such that $A = 100$ and
$C = 100$ and that the two poles of height $h$ are placed at $x = -50m$
and $x = +50m$ as the graph shows. Then assume that the trees we
are dealing with can reach a maximum height of $10m$.
What pole height h is needed such that the cable just clears $10m$?
Use the following value: $\cosh(0.5) \approx1.13$ (no calculator needed!).
	
	\vspace{9 cm}
	\textbf{Bonus:} Completed the work early? Calculate the length of cable for the above problem, using the following formula for
cable length.
\[L(x)=\int_a^b \sqrt{1+(f'(x))^2} dx\]
where $f(x)$ is the hyperbolic function provided in the previous example:
$f(x) = 100 \cosh(x/100) + B.$ Use the following
value: $\sinh(0.5)\approx 0.52$ (no calculator needed!).
\end{document}