\documentclass[10pt]{article}

% Set the margins to be something normal.
\usepackage[margin=1in]{geometry}

% Fun lists!
\usepackage{enumitem}
\usepackage{textcomp}
\setitemize{label=\textrightarrow, itemsep=0pt}

% Math symbols.
\usepackage{amsmath, amssymb, amsfonts}
\usepackage{tikz}

\makeatletter
\newcommand\cdotfill{%
    \leavevmode\cleaders\hb@xt@.44em{\hss$\cdot$\hss}\hfill\kern\z@
}

% No indents.
\setlength\parindent{0pt}

% Nice fonts :)
\usepackage{tgschola}

\begin{document}
	\pagenumbering{gobble}
	{
		\centering
		\huge{Week 15 --- Semester Overview} \\
		\Large{MATH:114, Recitations 309 and 310} \\[2em]
	}
	\cdotfill \ \ \ \textbf{Midterm 1} \ \ \ \cdotfill
	
	\vspace{2em}
	1. How can we use integrals to find the \textit{area between two curves}? How can we use geometry to estimate these calculations?
	
	\vspace{0.1\textheight}
	2. Describe how the \textit{shell} and \textit{washer} methods work for finding volumes of solids of rotation. What geometric ideas are at play?
	
	\vspace{0.1\textheight}
	3. Describe the ideas behind computing \textit{curve lengths} and \textit{surface} areas of solids of rotation. How are these related, if at all, to the shell and washer methods?
	
	\vspace{0.1\textheight}
	4. What does \textit{Euler's formula} tell us?
	
	\vspace{0.1\textheight}
	\cdotfill \ \ \ \textbf{Midterm 2} \ \ \ \cdotfill
	
	\vspace{2em}
	5. Talk about the ``information'' included in \textit{linear} and \textit{quadratic} approximations. How are these approximations related to \textit{Taylor polynomials}?
	
	\vspace{0.1\textheight}
	6. Why do we use \textit{trigonometric substitutions} when integrating? What theorems and identities make these substitutions work?
	
	\newpage
	7. What is \textit{integration by parts}? How does it work?
	
	
	\vspace{0.1\textheight}
	8. What makes \textit{partial fraction decomposition} a useful tool?
	
	\vspace{0.1\textheight}
	9. Why do we use \textit{improper integrals}? Equivalently, what problem does an improper integral help us avoid?
	
	\vspace{0.1\textheight}
	10. Discuss the ideas behind the three \textit{numerical integration} techniques: the \textit{midpoint rule}, the \textit{trapezoid rule}, and \textit{Simpson's rule}. 
	
	\vspace{0.1\textheight}
	11. What does it mean for an improper integral to \textit{converge} or \textit{diverge}? What techniques can we use to tell whether an improper integral converges or diverges?
	
	\vspace{0.1\textheight}
	\cdotfill \ \ \ \textbf{Midterm 3} \ \ \ \cdotfill
	
	\vspace{2em}
	12. What is a \textit{sequence}? What does it mean for a sequence to \textit{converge}? Can you express this idea mathematically?
	
	\vspace{0.1\textheight}
	13. What is a \textit{series}? What is the relationship between series and sequences? If a series converges, what can we say about its underlying sequence, and its sequence of partial sums?
	
	\vspace{0.1\textheight}
	14. What tests can we use to determine whether a series converges or diverges? \textit{Why} do they work?
\end{document}
